\input{Algo1Macros}
\usepackage{caratula}
\usepackage{enumerate}
\decimalpoint

\setlength{\parindent}{0em}
\setlength{\parskip}{0.5em}

\begin{document}

\titulo{Práctica 3}
\fecha{2do cuatrimestre 2021 (virtual)}
\materia{Algoritmos y Estructuras de Datos 1}
\integrante{Jonathan Bekenstein}{348/11}{jbekenstein@dc.uba.ar}

\maketitle

\section{Práctica 3}

\subsection{Ejercicio 1}

\subsubsection{Pregunta A}

El problema es que la post condición se puede indefinir si result está fuera del rango de la secuenca. Y eso no puede suceder nunca, las pre y post condiciones solo pueden ser verdaderas o falsas, nunca indefinidas.

\begin{proc}{buscar}{\In l: \TLista{\float}, \In elem: \float, \Out result: \ent}{}
    \pre{elem \in l}
    \post{0 \leq result < |l| \yLuego l[result] = elem}
\end{proc}

\subsubsection{Pregunta B}

El problema es que se indefine al indexar $l[i-1]$ cuando $i=0$. Como queremos verificar que el elemento en el índice $i$ sea el doble que el elemento en el índice $i-1$, tenemos que arrancar a revisar desde $i=1$. Si la secuencia tiene un único elemento, entonces no hay que revisar nada pues el primer número de la progresión geométrica no va a ser el doble de nadie.

\begin{proc}{progresionGeometricaFactor2}{\In l: \TLista{\ent}, \Out result: \bool}{}
    \pre{True}
    \post{result = True \leftrightarrow ((\forall i: \ent)(1 \leq i < |l| \implicaLuego l[i] = 2 * l[i-1]))}
\end{proc}

\subsubsection{Pregunta C}

El problema es que en la post condición se pide $y \neq x$ pero en el contexto de esta especificación, $x$ no está definido. En cambio, lo que habría que pedir es que $y \neq result$ o más simple aún, quitar esa condición y pedir $y \geq result$. A su vez, también falta especificar que $result \in l$ para garantizar que $result$ realmente sea un elemento de la secuencia.

\begin{proc}{minimo}{\In l: \TLista{\ent}, \Out result: \ent}{}
    \pre{True}
    \post{result \in l \land (\forall y: \ent)(y \in l \rightarrow y \geq result)}
\end{proc}

\subsection{Ejercicio 2}

\subsubsection{Pregunta A}

Por ejemplo $l = \langle 1 \rangle$, $suma = 2$. Cumplen la pre condición que es simplemente $True$ (o sea, cualquiera cosa cumple la pre condición). Pero no existe forma de cumplir con la post condición ya que no hay suficientes elementos en $l$ para que sumados den $2$.

\subsubsection{Pregunta B}

Sigue siendo inválida porque solo restringe el valor máximo y mínimo que puede tener $suma$ pero no garantiza que efectivamente existan elementos en $l$ que sumados den $suma$. Por ejemplo $l = \langle 1, 3 \rangle$, $suma = 2$. Con estos valores se cumple la pre condición: $min\_suma(l) \leq suma \leq max\_suma(l) \leftrightarrow 0 \leq 2 \leq 3$ pero no existen elementos en $l$ que sumados den exactamente $2$.

\subsubsection{Pregunta C}

$(\exists s: \TLista{\ent})( (\forall x: \ent)( \#apariciones(x, s) \leq \#apariciones(x, l) ) \land suma = \sum_{i=0}^{|s|-1} s[i] )$

\subsection{Ejercicio 3}

\subsubsection{Pregunta A}

\begin{enumerate}[I)]
    \item $x = 0 \rightarrow result \in \{ 0 \}$
    \item $x = 1 \rightarrow result \in \{ -1, 1 \}$
    \item $x = 27 \rightarrow result \in \{ -\sqrt{27}, \sqrt{27} \}$
\end{enumerate}

\subsubsection{Pregunta B}

\begin{enumerate}[I)]
    \item $l = \langle 1, 2, 3, 4 \rangle \rightarrow result \in \{ 3 \}$
    \item $l = \langle 15.5, -18, 4.215, 15.5, -1 \rangle \rightarrow result \in \{ 0, 3 \}$
    \item $l = \langle 0, 0, 0, 0, 0, 0 \rangle \rightarrow result \in \{ 0, 1, 2, 3, 4, 5 \}$
\end{enumerate}

\subsubsection{Pregunta C}

\begin{enumerate}[I)]
    \item $l = \langle 1, 2, 3, 4 \rangle \rightarrow result = 3$
    \item $l = \langle 15.5, -18, 4.215, 15.5, -1 \rangle \rightarrow result = 0$
    \item $l = \langle 0, 0, 0, 0, 0, 0 \rangle \rightarrow result = 0$
\end{enumerate}

\subsubsection{Pregunta D}

$indiceDelPrimerMaximo$ y $indiceDelMaximo$ tienen necesariamente la misma salida cuando no hay valores repetidos en la secuencia $l$. En estos casos, sería cuando $l = \langle 1, 2, 3, 4 \rangle$.

\subsection{Ejercicio 4}

\subsubsection{Pregunta A}

Incorrecta porque las 2 expresiones deberían estar unidas con un $\lor$, ya que sino es imposible que se cumplan ambas al mismo tiempo (pues piden $a < 0$ y también $a \geq 0$).

\subsubsection{Pregunta B}

Incorrecta porque la post condición no contempla el caso cuando $a = 0$.

\subsubsection{Pregunta C}

Correcta.

\subsubsection{Pregunta D}

Correcta.

\subsubsection{Pregunta E}

Incorrecta porque cuando $a \geq 0$, la implicación $a < 0 \rightarrow result = 2 * b$ resulta $True$ pues no se cumple el antecedente. Y luego como las 2 implicaciones están unidas con un $\lor$, este $True$ ya hace que toda la post condición sea $True$ sin importar si efectivamente $result = b - 1$ como debería ser según la especificación. Pasa lo mismo de forma análoga cuando $a < 0$.

\subsubsection{Pregunta F}

Correcta.

\end{document}
