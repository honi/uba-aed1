\input{Algo1Macros}
\usepackage{caratula}
\usepackage{enumerate}
\usepackage{hyperref}

\decimalpoint
\hypersetup{colorlinks=true, linkcolor=black, urlcolor=blue}
\setlength{\parindent}{0em}
\setlength{\parskip}{0.5em}
\setcounter{tocdepth}{2}
\setcounter{section}{2}
\renewcommand{\thesubsubsection}{\thesubsection.\Alph{subsubsection}}

\begin{document}

\titulo{Práctica 3}
\fecha{2do cuatrimestre 2021 (virtual)}
\materia{Algoritmos y Estructuras de Datos 1}
\integrante{Jonathan Bekenstein}{348/11}{jbekenstein@dc.uba.ar}

\maketitle

\tableofcontents
\newpage

\section{Práctica 3}

\subsection{Ejercicio 1}

\subsubsection{Pregunta A}

El problema es que la postcondición se puede indefinir si result está fuera del rango de la secuenca. Y eso no puede suceder nunca, las pre y post condiciones solo pueden ser verdaderas o falsas, nunca indefinidas.

\begin{proc}{buscar}{\In l: \TLista{\float}, \In elem: \float, \Out result: \ent}{}
    \pre{elem \in l}
    \post{0 \leq result < |l| \yLuego l[result] = elem}
\end{proc}

\subsubsection{Pregunta B}

El problema es que se indefine al indexar $l[i-1]$ cuando $i=0$. Como queremos verificar que el elemento en el índice $i$ sea el doble que el elemento en el índice $i-1$, tenemos que arrancar a revisar desde $i=1$. Si la secuencia tiene un único elemento, entonces no hay que revisar nada pues el primer número de la progresión geométrica no va a ser el doble de nadie.

\begin{proc}{progresionGeometricaFactor2}{\In l: \TLista{\ent}, \Out result: \bool}{}
    \pre{True}
    \post{result = True \leftrightarrow ((\forall i: \ent)(1 \leq i < |l| \implicaLuego l[i] = 2 * l[i-1]))}
\end{proc}

\subsubsection{Pregunta C}

El problema es que en la postcondición se pide $y \neq x$ pero en el contexto de esta especificación, $x$ no está definido. En cambio, lo que habría que pedir es que $y \neq result$ o más simple aún, quitar esa condición y pedir $y \geq result$. A su vez, también falta especificar que $result \in l$ para garantizar que $result$ realmente sea un elemento de la secuencia.

\begin{proc}{minimo}{\In l: \TLista{\ent}, \Out result: \ent}{}
    \pre{True}
    \post{result \in l \land (\forall y: \ent)(y \in l \rightarrow y \geq result)}
\end{proc}

\subsection{Ejercicio 2}

\subsubsection{Pregunta A}

Por ejemplo $l = \langle 1 \rangle$, $suma = 2$. Cumplen la precondición que es simplemente $True$ (o sea, cualquiera cosa cumple la precondición). Pero no existe forma de cumplir con la postcondición ya que no hay suficientes elementos en $l$ para que sumados den $2$.

\subsubsection{Pregunta B}

Sigue siendo inválida porque solo restringe el valor máximo y mínimo que puede tener $suma$ pero no garantiza que efectivamente existan elementos en $l$ que sumados den $suma$. Por ejemplo $l = \langle 1, 3 \rangle$, $suma = 2$. Con estos valores se cumple la precondición: $min\_suma(l) \leq suma \leq max\_suma(l) \leftrightarrow 0 \leq 2 \leq 3$ pero no existen elementos en $l$ que sumados den exactamente $2$.

\subsubsection{Pregunta C}

$(\exists s: \TLista{\ent})( (\forall x: \ent)( \#apariciones(x, s) \leq \#apariciones(x, l) ) \land suma = \sum_{i=0}^{|s|-1} s[i] )$

\subsection{Ejercicio 3}

\subsubsection{Pregunta A}

\begin{enumerate}[I)]
    \item $x = 0 \rightarrow result \in \{ 0 \}$
    \item $x = 1 \rightarrow result \in \{ -1, 1 \}$
    \item $x = 27 \rightarrow result \in \{ -\sqrt{27}, \sqrt{27} \}$
\end{enumerate}

\subsubsection{Pregunta B}

\begin{enumerate}[I)]
    \item $l = \langle 1, 2, 3, 4 \rangle \rightarrow result \in \{ 3 \}$
    \item $l = \langle 15.5, -18, 4.215, 15.5, -1 \rangle \rightarrow result \in \{ 0, 3 \}$
    \item $l = \langle 0, 0, 0, 0, 0, 0 \rangle \rightarrow result \in \{ 0, 1, 2, 3, 4, 5 \}$
\end{enumerate}

\subsubsection{Pregunta C}

\begin{enumerate}[I)]
    \item $l = \langle 1, 2, 3, 4 \rangle \rightarrow result = 3$
    \item $l = \langle 15.5, -18, 4.215, 15.5, -1 \rangle \rightarrow result = 0$
    \item $l = \langle 0, 0, 0, 0, 0, 0 \rangle \rightarrow result = 0$
\end{enumerate}

\subsubsection{Pregunta D}

$indiceDelPrimerMaximo$ y $indiceDelMaximo$ tienen necesariamente la misma salida cuando no hay valores repetidos en la secuencia $l$. En estos casos, sería cuando $l = \langle 1, 2, 3, 4 \rangle$.

\subsection{Ejercicio 4}

\subsubsection{Pregunta A}

Incorrecta porque las 2 expresiones deberían estar unidas con un $\lor$, ya que sino es imposible que se cumplan ambas al mismo tiempo (pues piden $a < 0$ y también $a \geq 0$).

\subsubsection{Pregunta B}

Incorrecta porque la postcondición no contempla el caso cuando $a = 0$.

\subsubsection{Pregunta C}

Correcta.

\subsubsection{Pregunta D}

Correcta.

\subsubsection{Pregunta E}

Incorrecta porque cuando $a \geq 0$, la implicación $a < 0 \rightarrow result = 2 * b$ resulta $True$ pues no se cumple el antecedente. Y luego como las 2 implicaciones están unidas con un $\lor$, este $True$ ya hace que toda la postcondición sea $True$ sin importar si efectivamente $result = b - 1$ como debería ser según la especificación. Pasa lo mismo de forma análoga cuando $a < 0$.

\subsubsection{Pregunta F}

Correcta.

\subsection{Ejercicio 5}

\subsubsection{Pregunta A}

Si recibe $x = 3$ devuelve $result = 9$, lo cual hace verdadera la postcondición pues $9 > 3$.

\subsubsection{Pregunta B}

$x = 0.5 \rightarrow result = 0.5^2 = 0.25 \ngtr 0.5$

$x = 1 \rightarrow result = 1^2 = 1 \ngtr 1$

$x = -0.2 \rightarrow result = (-0.2)^2 = 0.04 > -0.2$

$x = -7 \rightarrow result = (-7)^2 = 49 > -7$

\subsubsection{Pregunta C}

\begin{proc}{unoMasGrande}{\In x: \float, \Out result: \float}{}
    \pre{x < 0 \lor x > 1}
    \post{result > x}
\end{proc}

\subsection{Ejercicio 6}

\subsubsection{Pregunta A}

$P3 > P1 > P2$

\subsubsection{Pregunta B}

$Q3 > Q1 > Q2$

\subsubsection{Pregunta C}

Programa 1: $r := x * x$

Programa 2: $r := x * x + 1$

\subsubsection{Pregunta D}

\begin{enumerate}[a)]
    \item Cumple porque la nueva precondición (P3) es más fuerte que la precondición original (P1).
    \item No cumple porque la nueva precondición (P2) es más débil que la precondición original (P1).
    \item Cumple porque la nueva postcondición (Q2) es más débil que la postcondición original (Q1).
    \item No cumple porque la nueva postcondición (Q3) es más fuerte que la postcondición original (Q1).
    \item Cumple porque la nueva precondición (P3) es más fuerte que la precondición original (P1) y la nueva postcondición (Q2) es más débil que la postcondición original (Q1).
    \item No cumple porque la nueva precondición (P2) es más débil que la precondición original (P1).
    \item No cumple porque la nueva postcondición (Q3) es más fuerte que la postcondición original (Q1).
    \item No cumple porque la nueva precondición (P2) es más débil que la precondición original (P1) y además la nueva postcondición (Q3) es más fuerte que la postcondición original (Q1).
\end{enumerate}

\subsubsection{Pregunta E}

Dado un algoritmo que cumple con una especificación, es posible reemplazar dicha especificación por otra y que el algoritmo siga cumpliendo si:

\begin{enumerate}[1)]
    \item La nueva precondición es más fuerte que la original y la nueva postcondición es más débil que la original.
    \item La nueva precondición es más fuerte que la original y la postcondición se mantiene igual
    \item La precondición se mantiene igual y la nueva postcondición es más débil que la original.
\end{enumerate}

\subsection{Ejercicio 7}

\subsubsection{Pregunta A}

Sabiendo que vale la precondición de $p1$ se puede afirmar que $x \neq 0$.

Luego, se puede dividir en 2 casos para ver cuándo vale la precondición de $p2$:

\begin{enumerate}[1)]
    \item Si $n > 0$ el antecedente de la implicación es falso y así la implicación resulta verdadera, sin importar el valor de $x$.
    \item Si $n \leq 0$ la implicación resulta verdadera si $x \neq 0$. Esto vale pues sabemos que se cumple la precondición de $p1$.
\end{enumerate}

Por lo tanto vale la precondición de $p2$.

\subsubsection{Pregunta B}

En esencia lo que me piden es probar que $Post_{p2} \rightarrow Post_{p1} \equiv (result = [x^n] \rightarrow x^n - 1 < result \leq x^n)$.

Esto depende del algoritmo usado para calcular la parte entera de $x^n$. Si se usa la función \href{https://es.wikipedia.org/wiki/Funciones_de_parte_entera#Funci%C3%B3n_techo}{techo}, entonces la implicación vale pues $Post_{p2}$ es literalmente la definición de esa función. Pero si se usa otro algoritmo, por ejemplo la función \href{https://es.wikipedia.org/wiki/Funciones_de_parte_entera#Funci%C3%B3n_piso/suelo}{piso}, entonces la implicación no siempre vale.

\subsubsection{Pregunta C}

No necesariamente, depende del algoritmo usado para calcular la parte entera de $x^n$.

\subsection{Ejercicio 8}

Notar que $Pre_{n-esimo1}$ compara con $<$, lo cual significa que no pueden haber 2 elementos iguales en la secuencia $l$. Por lo tanto, vale que $Pre_{n-esimo1} \rightarrow Pre_{n-esimo2}$.

Por otro lado, $Post_{n-esimo1}$ nos dice que $result \in l $ y además que está en la posicón $n$. Debido a que $Pre_{n-esimo1}$ garantiza que la secuencia $l$ está ordenada, la forma de obtener el índice de $result$ definida en $Post_{n-esimo2}$ en efecto nos va a dar el valor correcto para $n$.

Al revés no funciona porque $Pre_{n-esimo2}$ solo garantiza que los elementos de la secuencia $l$ sean distintos entre sí, pero eso no implica que la secuencia esté ordenada. Por ejemplo $\langle 1,3,2 \rangle$ satisface $Pre_{n-esimo2}$ pero no $Pre_{n-esimo1}$.

\subsection{Ejercicio 9}

\subsubsection{Pregunta A}

Dado un número entero, decidir si es par.

\begin{proc}{esPar}{\In n: \ent, \Out r: \bool}{}
    \pre{True}
    \post{r = True \leftrightarrow n \bmod 2 = 0}
\end{proc}

\subsubsection{Pregunta B}

Dado un entero $n$ y uno $m$, decidir si $n$ es un múltiplo de $m$.

\begin{proc}{esMúltiplo}{\In n: \ent, \In m: \ent, \Out r: \bool}{}
    \pre{True}
    \post{r = True \leftrightarrow n \bmod m = 0}
\end{proc}

\subsubsection{Pregunta C}

Dado un número real, devolver su inverso multiplicativo.

\begin{proc}{inversoMultiplicativo}{\In x: \float, \Out r: \float}{}
    \pre{x \neq 0}
    \post{r = 1 / x}
\end{proc}

\subsubsection{Pregunta D}

Dada una secuencia de caracteres, obtener de ella solo los que son numéricos (con todas sus apariciones sin importar el orden de aparición).

\begin{proc}{subseqDeNumeros}{\In s: \TLista{Char}, \Out r: \TLista{Char}}{}
    \pre{True}
    \post{(\forall c: Char)(\#apariciones(r, c) = \IfThenElse{`0` \leq c \leq `9`}{\#apariciones(s, c)}{0})}
\end{proc}

\subsubsection{Pregunta E}

Dada una secuencia de reales, devolver la secuencia que resulta de duplicar sus valores en las posiciones impares.

\begin{proc}{duplicarPosicionesImpares}{\In s: \TLista{\float}, \Out r: \TLista{\float}}{}
    \pre{True}
    \post{|r| = |s| \land (\forall i: \ent)(0 \leq i < |r| \implicaLuego r[i] = \IfThenElse{i \bmod 2 = 0}{s[i]}{s[i] * 2})}
\end{proc}

\subsubsection{Pregunta F}

Dado un número entero, listar todos sus divisores positivos (sin duplicados).

\begin{proc}{divisoresPositivos}{\In n: \ent, \Out r: \TLista{\ent}}{}
    \pre{True}
    \post{(\forall k: \ent)((k > 0 \land n \bmod k = 0 \leftrightarrow k \in r) \land (\#apariciones(r, k) \leq 1))}
\end{proc}

\subsection{Ejercicio 10}

\subsubsection{Pregunta A}

Tiene sentido la pregunta y la respuesta es que no, $4$ no es múltiplo de $0$ pues $\nexists n \in \ent \mid 4 = 0 * n$.

\subsubsection{Pregunta B}

Sí, debería ser una entrada válida pero no lo es en la especificación dada, pues la precondición pide $m \neq 0$.

\subsubsection{Pregunta C}

\begin{proc}{esMultiplo]}{\In n, m: \ent, \Out result: \bool}{}
    \pre{True}
    \post{result = \IfThenElse{m = 0}{n = 0}{n \bmod m = 0}}
\end{proc}

\subsubsection{Pregunta D}

La precondición original es más fuerte pues $m \neq 0 \rightarrow True$ es una tautología.

\subsection{Ejercicio 11}

\subsubsection{Pregunta A}

A priori la especificación solo indica duplicar los valores en las posiciones impares, pero no dice nada explícito sobre qué hacer con los valores en las posiciones pares.

Si solo consideramos el requerimiento de duplicar los valores en las posiciones impares, entonces el resultado sería correcto.

Si además suponemos que se deben mantener intactos los valores en las posiciones pares, entonces el resultado no sería correcto.

Por eso, honrando al Zen de Python, explícito es mejor que implícito.

\subsubsection{Pregunta B}

Sí, satisface la postcondición.

\subsubsection{Pregunta C}

Asumo que ``el resultado esperado`` es que los valores en las posiciones pares sean exactamente los mismos valores de la secuencia de entrada $l$, además de duplicar los valores en las posiciones impares.

Esto ya lo implementé en el ejercicio 9.E.

\subsubsection{Pregunta D}

La nueva postcondición es más fuerte.

\subsection{Ejercicio 12}

\begin{proc}{enteroABinario}{\In n: \ent, \Out r: \TLista{\ent}}{}
    \pre{True}
    \post{(n = \sum_{i=0}^{|r|-1} r[|r| - 1 - i] * 2^i) \land (\forall i: \ent)(0 \leq i < |r| \implicaLuego (r[i] = 0 \lor r[i] = 1))}
\end{proc}

\subsection{Ejercicio 13}

Repaso:

\begin{enumerate}[1)]
    \item Sobreespecificado: postcondición mas restrictiva de lo necesario (excluye resultados válidos).
    \item Subespecificado: precondición mas restrictiva de lo necesario (excluye entradas válidas) o postcondición muy débil (incluye resultados inválidos).
\end{enumerate}

Entonces, respecto al ejercicio 8:

\begin{enumerate}[1)]
    \item n-esimo1 está subespecificado pues la precondición pide que la secuencia esté ordenada cuando eso no es realmente relevante al problema en cuestión, excluyendo así entradas válidas para el problema.
    \item n-esimo2 está sobreespecificado pues la postcondición implica que la secuencia está ordenada. Además eso tiene otro problema ya que la precondición no garantiza que la secuencia esté ordenada.
\end{enumerate}

{\em Nota: Verificar esta respuesta, no estoy seguro.}

\end{document}
